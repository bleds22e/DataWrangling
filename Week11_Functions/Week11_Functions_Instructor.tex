% Options for packages loaded elsewhere
\PassOptionsToPackage{unicode}{hyperref}
\PassOptionsToPackage{hyphens}{url}
%
\documentclass[
]{article}
\usepackage{amsmath,amssymb}
\usepackage{iftex}
\ifPDFTeX
  \usepackage[T1]{fontenc}
  \usepackage[utf8]{inputenc}
  \usepackage{textcomp} % provide euro and other symbols
\else % if luatex or xetex
  \usepackage{unicode-math} % this also loads fontspec
  \defaultfontfeatures{Scale=MatchLowercase}
  \defaultfontfeatures[\rmfamily]{Ligatures=TeX,Scale=1}
\fi
\usepackage{lmodern}
\ifPDFTeX\else
  % xetex/luatex font selection
\fi
% Use upquote if available, for straight quotes in verbatim environments
\IfFileExists{upquote.sty}{\usepackage{upquote}}{}
\IfFileExists{microtype.sty}{% use microtype if available
  \usepackage[]{microtype}
  \UseMicrotypeSet[protrusion]{basicmath} % disable protrusion for tt fonts
}{}
\makeatletter
\@ifundefined{KOMAClassName}{% if non-KOMA class
  \IfFileExists{parskip.sty}{%
    \usepackage{parskip}
  }{% else
    \setlength{\parindent}{0pt}
    \setlength{\parskip}{6pt plus 2pt minus 1pt}}
}{% if KOMA class
  \KOMAoptions{parskip=half}}
\makeatother
\usepackage{xcolor}
\usepackage[margin=1in]{geometry}
\usepackage{color}
\usepackage{fancyvrb}
\newcommand{\VerbBar}{|}
\newcommand{\VERB}{\Verb[commandchars=\\\{\}]}
\DefineVerbatimEnvironment{Highlighting}{Verbatim}{commandchars=\\\{\}}
% Add ',fontsize=\small' for more characters per line
\usepackage{framed}
\definecolor{shadecolor}{RGB}{248,248,248}
\newenvironment{Shaded}{\begin{snugshade}}{\end{snugshade}}
\newcommand{\AlertTok}[1]{\textcolor[rgb]{0.94,0.16,0.16}{#1}}
\newcommand{\AnnotationTok}[1]{\textcolor[rgb]{0.56,0.35,0.01}{\textbf{\textit{#1}}}}
\newcommand{\AttributeTok}[1]{\textcolor[rgb]{0.13,0.29,0.53}{#1}}
\newcommand{\BaseNTok}[1]{\textcolor[rgb]{0.00,0.00,0.81}{#1}}
\newcommand{\BuiltInTok}[1]{#1}
\newcommand{\CharTok}[1]{\textcolor[rgb]{0.31,0.60,0.02}{#1}}
\newcommand{\CommentTok}[1]{\textcolor[rgb]{0.56,0.35,0.01}{\textit{#1}}}
\newcommand{\CommentVarTok}[1]{\textcolor[rgb]{0.56,0.35,0.01}{\textbf{\textit{#1}}}}
\newcommand{\ConstantTok}[1]{\textcolor[rgb]{0.56,0.35,0.01}{#1}}
\newcommand{\ControlFlowTok}[1]{\textcolor[rgb]{0.13,0.29,0.53}{\textbf{#1}}}
\newcommand{\DataTypeTok}[1]{\textcolor[rgb]{0.13,0.29,0.53}{#1}}
\newcommand{\DecValTok}[1]{\textcolor[rgb]{0.00,0.00,0.81}{#1}}
\newcommand{\DocumentationTok}[1]{\textcolor[rgb]{0.56,0.35,0.01}{\textbf{\textit{#1}}}}
\newcommand{\ErrorTok}[1]{\textcolor[rgb]{0.64,0.00,0.00}{\textbf{#1}}}
\newcommand{\ExtensionTok}[1]{#1}
\newcommand{\FloatTok}[1]{\textcolor[rgb]{0.00,0.00,0.81}{#1}}
\newcommand{\FunctionTok}[1]{\textcolor[rgb]{0.13,0.29,0.53}{\textbf{#1}}}
\newcommand{\ImportTok}[1]{#1}
\newcommand{\InformationTok}[1]{\textcolor[rgb]{0.56,0.35,0.01}{\textbf{\textit{#1}}}}
\newcommand{\KeywordTok}[1]{\textcolor[rgb]{0.13,0.29,0.53}{\textbf{#1}}}
\newcommand{\NormalTok}[1]{#1}
\newcommand{\OperatorTok}[1]{\textcolor[rgb]{0.81,0.36,0.00}{\textbf{#1}}}
\newcommand{\OtherTok}[1]{\textcolor[rgb]{0.56,0.35,0.01}{#1}}
\newcommand{\PreprocessorTok}[1]{\textcolor[rgb]{0.56,0.35,0.01}{\textit{#1}}}
\newcommand{\RegionMarkerTok}[1]{#1}
\newcommand{\SpecialCharTok}[1]{\textcolor[rgb]{0.81,0.36,0.00}{\textbf{#1}}}
\newcommand{\SpecialStringTok}[1]{\textcolor[rgb]{0.31,0.60,0.02}{#1}}
\newcommand{\StringTok}[1]{\textcolor[rgb]{0.31,0.60,0.02}{#1}}
\newcommand{\VariableTok}[1]{\textcolor[rgb]{0.00,0.00,0.00}{#1}}
\newcommand{\VerbatimStringTok}[1]{\textcolor[rgb]{0.31,0.60,0.02}{#1}}
\newcommand{\WarningTok}[1]{\textcolor[rgb]{0.56,0.35,0.01}{\textbf{\textit{#1}}}}
\usepackage{graphicx}
\makeatletter
\def\maxwidth{\ifdim\Gin@nat@width>\linewidth\linewidth\else\Gin@nat@width\fi}
\def\maxheight{\ifdim\Gin@nat@height>\textheight\textheight\else\Gin@nat@height\fi}
\makeatother
% Scale images if necessary, so that they will not overflow the page
% margins by default, and it is still possible to overwrite the defaults
% using explicit options in \includegraphics[width, height, ...]{}
\setkeys{Gin}{width=\maxwidth,height=\maxheight,keepaspectratio}
% Set default figure placement to htbp
\makeatletter
\def\fps@figure{htbp}
\makeatother
\setlength{\emergencystretch}{3em} % prevent overfull lines
\providecommand{\tightlist}{%
  \setlength{\itemsep}{0pt}\setlength{\parskip}{0pt}}
\setcounter{secnumdepth}{-\maxdimen} % remove section numbering
\ifLuaTeX
  \usepackage{selnolig}  % disable illegal ligatures
\fi
\usepackage{bookmark}
\IfFileExists{xurl.sty}{\usepackage{xurl}}{} % add URL line breaks if available
\urlstyle{same}
\hypersetup{
  pdftitle={Week 11: Writing Functions},
  pdfauthor={Ellen Bledsoe},
  hidelinks,
  pdfcreator={LaTeX via pandoc}}

\title{Week 11: Writing Functions}
\author{Ellen Bledsoe}
\date{2025-04-08}

\begin{document}
\maketitle

\section{Writing Our Own Functions}\label{writing-our-own-functions}

Functions are the foundation of programming in R. So far, we have been
using functions that other people have written for us. We all have the
ability to write our \emph{own} functions to use in R, though!

\subsubsection{Understandable and Reusable
Code}\label{understandable-and-reusable-code}

Writing our own functions can help us to write code in understandable
chunks, reduce errors, and also make more reusable code that we can
transfer within and between projects, ultimately increasing
productivity.

We often run into instances where we want to do something many times but
slightly differently.

If you find yourself needing to run the same calculations or make the
same plot repeatedly, writing a function to accomplish that task can
save you from copying and pasting the same code over and over
again\ldots and a lot of time de-bugging repetitive code.

\subsection{Function Structure}\label{function-structure}

When we create a function, the structure of the function generally looks
like this:

\includegraphics[width=0.75\textwidth,height=\textheight]{images/r-function-syntax.png}

A slightly different way to think about this is through inputs and
outputs.

\begin{Shaded}
\begin{Highlighting}[]
\CommentTok{\# function\_name \textless{}{-} function(inputs) \{}
\CommentTok{\#   output \textless{}{-} do\_something(inputs)}
\CommentTok{\#   return(output)}
\CommentTok{\# \}}
\end{Highlighting}
\end{Shaded}

The braces indicate that the lines of code are a group that gets run
together.

\begin{itemize}
\tightlist
\item
  A function will run all of the lines of code in the braces using the
  arguments provided.
\item
  We then need to return the output using the \texttt{return()}
  function.
\end{itemize}

Let's write out our first function. We want to calculate shrub volumes.

\begin{Shaded}
\begin{Highlighting}[]
\NormalTok{calc\_shrub\_vol }\OtherTok{\textless{}{-}} \ControlFlowTok{function}\NormalTok{(length, width, height) \{}
\NormalTok{  volume }\OtherTok{\textless{}{-}}\NormalTok{ length }\SpecialCharTok{*}\NormalTok{ width }\SpecialCharTok{*}\NormalTok{ height}
  \FunctionTok{return}\NormalTok{(volume)}
\NormalTok{\}}
\end{Highlighting}
\end{Shaded}

It is important to note that writing the function does not create the
function. We have to remember to run the code to create the function and
add it to our environment.

When we run this code chunk above, we can see that the function gets
added to our environment in a new section called ``functions.''

Let's use our new function.

\begin{Shaded}
\begin{Highlighting}[]
\FunctionTok{calc\_shrub\_vol}\NormalTok{(}\FloatTok{0.8}\NormalTok{, }\FloatTok{1.6}\NormalTok{, }\FloatTok{2.0}\NormalTok{)}
\end{Highlighting}
\end{Shaded}

\begin{verbatim}
## [1] 2.56
\end{verbatim}

As with other functions, we can store the output as an object to use
later.

\begin{Shaded}
\begin{Highlighting}[]
\NormalTok{shrub\_vol }\OtherTok{\textless{}{-}} \FunctionTok{calc\_shrub\_vol}\NormalTok{(}\FloatTok{0.8}\NormalTok{, }\FloatTok{1.6}\NormalTok{, }\FloatTok{2.0}\NormalTok{)}
\end{Highlighting}
\end{Shaded}

\paragraph{Let's Practice!}\label{lets-practice}

Work on Question 1 in the Assignment.

\subsection{Black Box}\label{black-box}

It is important to remember a few things about functions. In many ways,
we need to treat functions as a black box.

What we mean by this is that the only thing that the function knows
about are the inputs we give it.

Similarly, the only thing our environment know about the function is the
output that we ask the function to return to us.

So, in the function above, we can't ``access'' variables or arguments
that are created within the function. The function exists within it's
own environment, so to speak.

\begin{Shaded}
\begin{Highlighting}[]
\NormalTok{width}
\NormalTok{volume}
\end{Highlighting}
\end{Shaded}

\paragraph{Let's Practice!}\label{lets-practice-1}

Work on Question 2 in the Assignment.

\subsection{Default arguments}\label{default-arguments}

As with many of the functions that we have already used in this class,
we can set defaults for the arguments in the functions that we create.

For example, if many of our shrubs are the same height, we could specify
that the default height should be 1.

\begin{Shaded}
\begin{Highlighting}[]
\NormalTok{calc\_shrub\_vol }\OtherTok{\textless{}{-}} \ControlFlowTok{function}\NormalTok{(length, width, }\AttributeTok{height =} \DecValTok{1}\NormalTok{) \{}
\NormalTok{  area }\OtherTok{\textless{}{-}}\NormalTok{ length }\SpecialCharTok{*}\NormalTok{ width}
\NormalTok{  volume }\OtherTok{\textless{}{-}}\NormalTok{ area }\SpecialCharTok{*}\NormalTok{ height}
  \FunctionTok{return}\NormalTok{(volume)}
\NormalTok{\}}

\FunctionTok{calc\_shrub\_vol}\NormalTok{(}\FloatTok{0.8}\NormalTok{, }\FloatTok{1.6}\NormalTok{)}
\end{Highlighting}
\end{Shaded}

\begin{verbatim}
## [1] 1.28
\end{verbatim}

\begin{Shaded}
\begin{Highlighting}[]
\FunctionTok{calc\_shrub\_vol}\NormalTok{(}\FloatTok{0.8}\NormalTok{, }\FloatTok{1.6}\NormalTok{, }\FloatTok{2.0}\NormalTok{)}
\end{Highlighting}
\end{Shaded}

\begin{verbatim}
## [1] 2.56
\end{verbatim}

\begin{Shaded}
\begin{Highlighting}[]
\FunctionTok{calc\_shrub\_vol}\NormalTok{(}\AttributeTok{length =} \FloatTok{0.8}\NormalTok{, }\AttributeTok{width =} \FloatTok{1.6}\NormalTok{, }\AttributeTok{height =} \FloatTok{2.0}\NormalTok{)}
\end{Highlighting}
\end{Shaded}

\begin{verbatim}
## [1] 2.56
\end{verbatim}

\subsubsection{When do we use argument
names?}\label{when-do-we-use-argument-names}

While you don't always have to name the arguments you are specifying,
there are some general suggestions for best practices.

As we saw above, as long as you specify arguments in the correct order,
they do not need to be named. On the other hand, if you specify
arguments out of the order in which they are written into the function
or you are skipping some arguments (using the default) and then
specifying later arguments, it is necessary to name them.

\begin{Shaded}
\begin{Highlighting}[]
\FunctionTok{calc\_shrub\_vol}\NormalTok{(}\AttributeTok{height =} \FloatTok{2.0}\NormalTok{, }\AttributeTok{length =} \FloatTok{0.8}\NormalTok{, }\AttributeTok{width =} \FloatTok{1.6}\NormalTok{)}
\end{Highlighting}
\end{Shaded}

\begin{verbatim}
## [1] 2.56
\end{verbatim}

With some complicated functions, the order of the arguments is often
challenging to remember, so using names by default is a good idea.

Similarly, when there are lots of optional arguments, you will often
need to name them if you are using something other than the default.
This is standard convention.

\begin{Shaded}
\begin{Highlighting}[]
\FunctionTok{calc\_shrub\_vol}\NormalTok{(}\FloatTok{0.8}\NormalTok{, }\FloatTok{1.6}\NormalTok{, }\AttributeTok{height =} \FloatTok{2.0}\NormalTok{)}
\end{Highlighting}
\end{Shaded}

\begin{verbatim}
## [1] 2.56
\end{verbatim}

\subsection{Combining Functions}\label{combining-functions}

Again, as with other functions, we can combine functions of our own
creation together.

\begin{Shaded}
\begin{Highlighting}[]
\NormalTok{est\_shrub\_mass }\OtherTok{\textless{}{-}} \ControlFlowTok{function}\NormalTok{(volume)\{}
\NormalTok{  mass }\OtherTok{\textless{}{-}} \FloatTok{2.65} \SpecialCharTok{*}\NormalTok{ volume}\SpecialCharTok{\^{}}\FloatTok{0.9}
\NormalTok{\}}

\NormalTok{shrub\_volume }\OtherTok{\textless{}{-}} \FunctionTok{calc\_shrub\_vol}\NormalTok{(}\FloatTok{0.8}\NormalTok{, }\FloatTok{1.6}\NormalTok{, }\FloatTok{2.0}\NormalTok{)}
\NormalTok{shrub\_mass }\OtherTok{\textless{}{-}} \FunctionTok{est\_shrub\_mass}\NormalTok{(shrub\_volume)}
\end{Highlighting}
\end{Shaded}

We can also use pipes with our own functions. The output from the first
function becomes the first argument for the second function.

\begin{Shaded}
\begin{Highlighting}[]
\FunctionTok{library}\NormalTok{(dplyr)}
\end{Highlighting}
\end{Shaded}

\begin{verbatim}
## 
## Attaching package: 'dplyr'
\end{verbatim}

\begin{verbatim}
## The following objects are masked from 'package:stats':
## 
##     filter, lag
\end{verbatim}

\begin{verbatim}
## The following objects are masked from 'package:base':
## 
##     intersect, setdiff, setequal, union
\end{verbatim}

\begin{Shaded}
\begin{Highlighting}[]
\NormalTok{shrub\_mass }\OtherTok{\textless{}{-}} \FunctionTok{calc\_shrub\_vol}\NormalTok{(}\FloatTok{0.8}\NormalTok{, }\FloatTok{1.6}\NormalTok{, }\FloatTok{2.0}\NormalTok{) }\SpecialCharTok{\%\textgreater{}\%} 
  \FunctionTok{est\_shrub\_mass}\NormalTok{()}
\end{Highlighting}
\end{Shaded}

\subsection{\texorpdfstring{Using the \texttt{tidyverse} in
Functions}{Using the tidyverse in Functions}}\label{using-the-tidyverse-in-functions}

One of the really nice things about the \texttt{tidyverse} is that we
usually don't need to put columns in quotes.

This is because they use ``tidy evaluation,'' a special type of
non-standard evaluation. Basically, they do fancy things under the
surface to make them easier to work with.

This is useful for when \emph{using} functions from the
\texttt{tidyverse}, but it means that we need to add an extra step when
we use \texttt{tidyverse} functions within our own functions that we are
\emph{writing}.

If we try to use \texttt{tidyverse} functions the way we normally would
within our functions, they won't work.

First, let's load our packages and get some data from Palmer Penguins.

\begin{Shaded}
\begin{Highlighting}[]
\FunctionTok{library}\NormalTok{(ggplot2)}
\FunctionTok{library}\NormalTok{(palmerpenguins)}

\NormalTok{penguins }\OtherTok{\textless{}{-}}\NormalTok{ penguins}
\end{Highlighting}
\end{Shaded}

Now, let's write some code for a ggplot within a function as we normally
would.

\begin{Shaded}
\begin{Highlighting}[]
\NormalTok{make\_plot }\OtherTok{\textless{}{-}} \ControlFlowTok{function}\NormalTok{(df, column, label) \{}
  \FunctionTok{ggplot}\NormalTok{(}\AttributeTok{data =}\NormalTok{ df, }\AttributeTok{mapping =} \FunctionTok{aes}\NormalTok{(}\AttributeTok{x =}\NormalTok{ column)) }\SpecialCharTok{+}
    \FunctionTok{geom\_histogram}\NormalTok{() }\SpecialCharTok{+}
    \FunctionTok{xlab}\NormalTok{(label)}
\NormalTok{\}}

\FunctionTok{make\_plot}\NormalTok{(penguins, body\_mass\_g, }\StringTok{"Body Mass (g)"}\NormalTok{)}
\end{Highlighting}
\end{Shaded}

To fix this issue, we have to tell our code which inputs/arguments are
this special type of data variable.

We can do this by ``embracing'' them in double braces (e.g.,
\texttt{\{\{\ var\ \}\}}). This tells the function to treat them in the
``tidy evaluation'' method.

\begin{Shaded}
\begin{Highlighting}[]
\NormalTok{make\_plot }\OtherTok{\textless{}{-}} \ControlFlowTok{function}\NormalTok{(df, column, label) \{}
    \FunctionTok{ggplot}\NormalTok{(}\AttributeTok{data =}\NormalTok{ df, }\AttributeTok{mapping =} \FunctionTok{aes}\NormalTok{(}\AttributeTok{x =}\NormalTok{ \{\{ column \}\})) }\SpecialCharTok{+}
    \FunctionTok{geom\_histogram}\NormalTok{() }\SpecialCharTok{+}
    \FunctionTok{xlab}\NormalTok{(label)}
\NormalTok{\}}

\FunctionTok{make\_plot}\NormalTok{(penguins, body\_mass\_g, }\StringTok{"Body Mass (g)"}\NormalTok{)}
\end{Highlighting}
\end{Shaded}

\begin{verbatim}
## `stat_bin()` using `bins = 30`. Pick better value with `binwidth`.
\end{verbatim}

\begin{verbatim}
## Warning: Removed 2 rows containing non-finite outside the scale range
## (`stat_bin()`).
\end{verbatim}

\includegraphics{Week11_Functions_Instructor_files/figure-latex/unnamed-chunk-13-1.pdf}

\begin{Shaded}
\begin{Highlighting}[]
\FunctionTok{make\_plot}\NormalTok{(penguins, bill\_length\_mm, }\StringTok{"Bill Length (mm)"}\NormalTok{)}
\end{Highlighting}
\end{Shaded}

\begin{verbatim}
## `stat_bin()` using `bins = 30`. Pick better value with `binwidth`.
\end{verbatim}

\begin{verbatim}
## Warning: Removed 2 rows containing non-finite outside the scale range
## (`stat_bin()`).
\end{verbatim}

\includegraphics{Week11_Functions_Instructor_files/figure-latex/unnamed-chunk-13-2.pdf}

\subsection{Code Design with
Functions}\label{code-design-with-functions}

Functions let us break code up into logical chunks that can be
understood in isolation.

When you write functions, place them at the top of your code then call
them below.

The functions hold the details. You can heavily comment the code of your
functions, as well. The function calls will show you the outline of the
code execution.

\subsubsection{Example Structure}\label{example-structure}

Functions (at or near the top of the document)

\begin{Shaded}
\begin{Highlighting}[]
\NormalTok{clean\_data }\OtherTok{\textless{}{-}} \ControlFlowTok{function}\NormalTok{(data)\{}
  \CommentTok{\# this function does stuff}
  \FunctionTok{do\_stuff}\NormalTok{(data)}
\NormalTok{\}}

\NormalTok{process\_data }\OtherTok{\textless{}{-}} \ControlFlowTok{function}\NormalTok{(cleaned\_data)\{}
  \CommentTok{\# this function does dplyr stuff}
  \FunctionTok{do\_dplyr\_stuff}\NormalTok{(cleaned\_data)}
\NormalTok{\}}

\NormalTok{make\_graph }\OtherTok{\textless{}{-}} \ControlFlowTok{function}\NormalTok{(processed\_data)\{}
  \CommentTok{\# this function plots stuff}
  \FunctionTok{do\_ggplot\_stuff}\NormalTok{(processed\_data)}
\NormalTok{\}}
\end{Highlighting}
\end{Shaded}

Using the functions in a different code chunk.

\begin{Shaded}
\begin{Highlighting}[]
\NormalTok{raw\_data }\OtherTok{\textless{}{-}} \FunctionTok{read.csv}\NormalTok{(}\StringTok{\textquotesingle{}mydata.csv\textquotesingle{}}\NormalTok{)}
\NormalTok{cleaned\_data }\OtherTok{\textless{}{-}} \FunctionTok{clean\_data}\NormalTok{(raw\_data)}
\NormalTok{processed\_data }\OtherTok{\textless{}{-}} \FunctionTok{process\_data}\NormalTok{(cleaned\_data)}
\FunctionTok{make\_graph}\NormalTok{(processed\_data)}
\end{Highlighting}
\end{Shaded}

\subsubsection{Source}\label{source}

Alternatively, if you have a lot of custom functions, you might consider
saving them in their own R script.

You can then use the \texttt{source()} function at the beginning of your
document to point to that file, read the file, and then the functions
will be available for you to use in your current document.

\begin{Shaded}
\begin{Highlighting}[]
\CommentTok{\# Example}
\FunctionTok{source}\NormalTok{(}\StringTok{"functions.R"}\NormalTok{)}
\end{Highlighting}
\end{Shaded}


\end{document}
